\documentclass[a4paper, 20pt]{article}
\usepackage[utf8]{inputenc}
\usepackage{paralist}
\usepackage{jeffe,handout}
\usepackage{eqarrays}
\usepackage{tikz}
\usetikzlibrary{matrix}
\usetikzlibrary{arrows,shapes.gates.logic.US,shapes.gates.logic.IEC,calc}
\usepackage{hyperref}
\usepackage{aurical}
\usepackage[T1]{fontenc}

\def\lnot{\mathop{\sim}}
\def\implies{\mathop{\rightarrow}}
\def\xor{\mathop{\oplus}}
\def\iff{\mathop{\leftrightarrow}}

\begin{document}
\headers{CPSC 121}{ }{2019S}

\begin{center}
    \LARGE
    \textbf{HW 1}
    \\[1ex]
    \Large Due: Wednesday, May 15, 2019 at 23:00 \\
\end{center}
    \LARGE
\begin{tabular}{rl}
 & \\
CSID 1: & d6a2b\\
 & \\
CSID 2: & b9i2b\\
 & \\
\end{tabular}
\large

\textbf{Instructions:}
\begin{enumerate}
\item Do not change the problem statements we are giving you. Simply add your solutions by editing this latex document. 
\item If you need more space, add a page between the existing pages using the \texttt{\textbackslash newpage} command.
\item Where possible, include formatting to clearly distinguish your solutions from the given problem text (e.g. use a different font colour for your solutions).
\item Export the completed assignment as a PDF file for upload to gradescope.
\item On Gradescope, upload only \textbf{one} copy per partnership. (Instructions for uploading to Gradescope are posted on the assignments page of the course website.)
\item Late submissions will be accepted up to 24 hours past the deadline with a penalty of 20\% of the assignment's maximum value.
\end{enumerate}

\textbf{Academic Conduct:} 
I certify that my assignment follows the academic conduct rules for of CPSC 121 as outlined on the course website. As part of those rules, when collaborating with anyone outside my group, (1) I and my collaborators took no record but names away, and (2) after a suitable break, my group created the assignment I am submitting without help from anyone other than the course staff. \\

\textbf{Version history:}
\begin{itemize}
    \item 2019-05-08 16:15 -- parenthesis added in Q1a; Q3 gate limit relaxed slightly
    \item 2019-05-05 20:31 -- Initial version for release
\end{itemize}

\newpage
\begin{question}{1.}
\def\StartsWith#1#2{\textit{Starts}({#1},{#2})}

\item[9] Consider the following sets and predicates: 
  \begin{itemize}
  \item $D$: video game developers
  \item $G$: video game titles
  \item $P$: hardware platforms
  \item $M(d, g)$: developer $d$ made game $g$.
  \item $R(g, p)$: game $g$ released on platform $p$.
  \item $B(d, g)$: developer $d$ bribed a journalist to write a favourable review for game $g$.
  \end{itemize}

  Rewrite each of  the following statements using \textbf{only}  the quantifiers $\forall$
  and $\exists$, the predicates $M$, $R$ and $B$, the domains $D$, $G$, $P$, and $\mathbb{R}$ (the set of real numbers),
  logical connectives, and the operators $=$, $\neq$. $<$, $\le$, $\ge$ and $>$.
  
  If you feel a  sentence is ambiguous, then state your assumed interpretation.
  \begin{question}{a.}[2]

  \item[3] Every developer made a game that released on every platform.
   
   \vspace{2.0in}
   
  \item[3] Some developer bribed a journalist to write a favourable review for a game made by a different developer.
  
  \vspace{2.0in}
   
  \item[3] Some developer bribed journalists to write favourable reviews for every game that the developer made.
  
  \vspace{2.0in}

  \end{question}

\newpage
\\1(a).
\\ Proof:
\\ $\equiv \lnot p \land q \land (\lnot s \lor (\lnot r \land s)) \lor (q \land(p \lor (\lnot p \land r \land s)))$ [Start]
\\ $\equiv \lnot p \land q \land (\lnot s \lor (\lnot r \land s)) \lor (q \land((p \lor \lnot p) \land(p \lor r) \land(p \lor s)))$ [DIST]
\\ $\equiv (\lnot p \land q \land ((\lnot s \lor \lnot r) \land (\lnot s \lor s))) \lor (q \land((p \lor \lnot p) \land(p \lor r) \land(p \lor s)))$ [DIST]
\\ $\equiv (\lnot p \land q \land ((\lnot s \lor \lnot r) \land T)) \lor (q \land((p \lor \lnot p) \land(p \lor r) \land(p \lor s)))$ [NEG]
\\ $\equiv (\lnot p \land q \land ((\lnot s \lor \lnot r) \land T)) \lor (q \land(T \land(p \lor r) \land(p \lor s)))$ [NEG]
\\ $\equiv (\lnot p \land q \land (\lnot s \lor \lnot r)) \lor (q \land(T \land(p \lor r) \land(p \lor s)))$ [I]
\\ $\equiv (\lnot p \land q \land (\lnot s \lor \lnot r)) \lor (q \land(p \lor r) \land(p \lor s))$ [I]
\\ $\equiv (q \land \lnot p \land (\lnot s \lor \lnot r)) \lor (q \land(p \lor r) \land(p \lor s))$ [COM]
\\ $\equiv q \land ((\lnot p \land (\lnot s \lor \lnot r)) \lor ((p \lor r) \land(p \lor s)))$ [DIST]
\\ $\equiv q \land ((\lnot p \land (\lnot s \lor \lnot r)) \lor (p \lor (r \land s)))$ [DIST]
\\ $\equiv q \land ((\lnot p \land \lnot (s \land r)) \lor (p \lor (r \land s)))$ [DM]
\\ $\equiv q \land (\lnot (p \lor (s \land r)) \lor (p \lor (r \land s)))$ [DM]
\\ $\equiv q \land (\lnot (p \lor (r \land s)) \lor (p \lor (r \land s)))$ [COM]
\\ $\equiv q \land T$ [NEG]
\\ $\equiv q$ [I]
\\ \boxed{}
\\
\\
\\1(b).
\\ Proof:
\\ $\equiv ((w \land x)\lor(\lnot w \land \lnot x))\land((\lnot y \land \lnot z)\lor(y \land z))\lor \lnot(w \land y)$ [Start]
\\ $\equiv (((w \land x)\lor(\lnot w \land \lnot x))\land((\lnot y \land \lnot z)\lor(y \land z)))\lor \lnot w \lor \lnot y$ [DM]
\\ $\equiv (((w \land x)\lor(\lnot w \land \lnot x))\lor \lnot w \land((\lnot y \land \lnot z)\lor(y \land z))\lor \lnot w ) \lor \lnot y$ [DIST]
\\ $\equiv (((w \land x)\lor(\lnot w \land \lnot x)\lor \lnot w) \land((\lnot y \land \lnot z)\lor(y \land z))\lor \lnot w ) \lor \lnot y$ [ASS]
\\ $\equiv (((w \land x)\lor \lnot w \lor (\lnot w \land \lnot x)) \land((\lnot y \land \lnot z)\lor(y \land z))\lor \lnot w ) \lor \lnot y$ [COM]
\\ $\equiv ((((w \lor \lnot w) \land (x \lor \lnot w)) \lor (\lnot w \land \lnot x)) \land((\lnot y \land \lnot z)\lor(y \land z))\lor \lnot w ) \lor \lnot y$ [DIST]
\\ $\equiv (((T \land (x \lor \lnot w)) \lor (\lnot w \land \lnot x)) \land((\lnot y \land \lnot z)\lor(y \land z))\lor \lnot w ) \lor \lnot y$ [NEG]
\\ $\equiv ((((x \lor \lnot w)) \lor (\lnot w \land \lnot x)) \land((\lnot y \land \lnot z)\lor(y \land z))\lor \lnot w ) \lor \lnot y$ [I]
\\ $\equiv ((((\lnot w \lor x)) \lor (\lnot w \land \lnot x)) \land((\lnot y \land \lnot z)\lor(y \land z))\lor \lnot w ) \lor \lnot y$ [COM]
\\ $\equiv ((\lnot w \lor x \lor (\lnot w \land \lnot x)) \land((\lnot y \land \lnot z)\lor(y \land z))\lor \lnot w ) \lor \lnot y$ [ASS]
\\ $\equiv ((\lnot w \lor (x \lor \lnot w \land x \lor \lnot x)) \land((\lnot y \land \lnot z)\lor(y \land z))\lor \lnot w ) \lor \lnot y$ [DIST]
\\ $\equiv ((\lnot w \lor (x \lor \lnot w \land T)) \land((\lnot y \land \lnot z)\lor(y \land z))\lor \lnot w ) \lor \lnot y$ [NEG]
\\ $\equiv ((\lnot w \lor (x \lor \lnot w)) \land((\lnot y \land \lnot z)\lor(y \land z))\lor \lnot w ) \lor \lnot y$ [I]
\\ $\equiv ((\lnot w \lor x \lor \lnot w) \land((\lnot y \land \lnot z)\lor(y \land z))\lor \lnot w ) \lor \lnot y$ [ASS]
\\ $\equiv ((\lnot w \lor \lnot w \lor x ) \land((\lnot y \land \lnot z)\lor(y \land z))\lor \lnot w ) \lor \lnot y$ [COM]
\\ $\equiv ((\lnot w \lor x ) \land((\lnot y \land \lnot z)\lor(y \land z))\lor \lnot w ) \lor \lnot y$ [ID]
\\ $\equiv ((\lnot w \lor x ) \land \lnot w \lor((\lnot y \land \lnot z)\lor(y \land z)) ) \lor \lnot y$ [COM]
\\ $\equiv \lnot w \lor( x \land((\lnot y \land \lnot z)\lor(y \land z)) ) \lor \lnot y$ [DIST]
\\ $\equiv \lnot w \lor(x \lor \lnot y \land((\lnot y \land \lnot z)\lor(y \land z)) \lor \lnot y ) $ [DIST]
\\ $\equiv \lnot w \lor((x \lor \lnot y) \land((\lnot y \land \lnot z)\lor(y \land z)) \lor \lnot y ) $ [ASS]
\\ $\equiv \lnot w \lor((x \lor \lnot y) \land((\lnot y \land \lnot z)\lor(y \land z) \lor \lnot y )) $ [ASS]
\\ $\equiv \lnot w \lor((x \lor \lnot y) \land((\lnot y \land \lnot z)\lor((y \lor \lnot y) \land (z \lor \lnot y )))) $ [DIST]
\\ $\equiv \lnot w \lor((x \lor \lnot y) \land((\lnot y \land \lnot z)\lor(T \land (z \lor \lnot y )))) $ [NEG]
\\ $\equiv \lnot w \lor((x \lor \lnot y) \land((\lnot y \land \lnot z)\lor((z \lor \lnot y )))) $ [I]
\\ $\equiv \lnot w \lor((x \lor \lnot y) \land((\lnot y \land \lnot z)\lor z \lor \lnot y )) $ [ASS]
\\ $\equiv \lnot w \lor((x \lor \lnot y) \land((\lnot y \lor z) \land (\lnot z \lor z) \lor \lnot y )) $ [ASS]
\\ $\equiv \lnot w \lor((x \lor \lnot y) \land((\lnot y \lor z) \land (\lnot z \lor z) \lor \lnot y )) $ [DIST]
\\ $\equiv \lnot w \lor((x \lor \lnot y) \land((\lnot y \lor z) \land T \lor \lnot y )) $ [NEG]
\\ $\equiv \lnot w \lor((x \lor \lnot y) \land((\lnot y \lor z)  \lor \lnot y )) $ [I]
\\ $\equiv \lnot w \lor((x \lor \lnot y) \land(\lnot y \lor z  \lor \lnot y )) $ [ASS]
\\ $\equiv \lnot w \lor((x \lor \lnot y) \land( z \lor \lnot y \lor \lnot y )) $ [COM]
\\ $\equiv \lnot w \lor((x \lor \lnot y) \land( z \lor \lnot y )) $ [COM]
\\ $\equiv \lnot w \lor(x  \land z)\lor \lnot y $ [DIST]
\\ \boxed{}
\newpage
%
% Question 2
%
\item[9] 

Using the definitions given in question 1, translate each of the
  following predicate logic statements into English. Try to make your English translations
  as natural sounding as possible.

  \begin{question}{a.}[3]
  \item[3] $\exists d \in D, \forall x,y \in G, \exists p_1, p_2 \in P, M(d,x) \land M(d,y) \land R(x,p_1) \land R(y,p_2) \implies p_1 = p_2 \implies \lnot(\exists g \in G, M(d,g) \land B(d,g))$.
  \begin{Questions}
    \vfill
  \end{Questions}
  
  \item[3] $\exists d \in D, \forall g \in G, \lnot M(d,g) \land (R(g, \text{Nintendo Switch}) \implies B(d,g))$.
  \begin{Questions}
    \vfill
  \end{Questions}

  \item[3] $\exists d \in D, \forall p \in P, \exists  g_1, g_2 \in G, M(d, g_1) \land M(d, g_2) \land R(g_1, p) \land \lnot R(g_2, p)$.
  \begin{Questions}
    \vfill
  \end{Questions}
  
  \end{question}
\\2(a). B \equiv D
\\ expression 1, expression 2, reason
\\$(\lnot a \lor b)\land(\lnot a \lor c)\land(\lnot a \lor \lnot d) \equiv a \implies (b \land c \land \lnot d)$ [Start]
\\$(\lnot a \lor b)\land(\lnot a \lor c)\land(\lnot a \lor \lnot d) \equiv \lnot a \lor (b \land c \land \lnot d)$ [IMP]
\\$\lnot a \lor (b \land c \land \lnot d) \equiv \lnot a \lor (b \land c \land \lnot d)$ [DIST]
\\
\\2(b). A \equiv E
\\ expression 1, expression 2, reason
\\$ (\lnot a \land b) \iff \lnot(c \lor d) \equiv \lnot ((a \lor \lnot b \lor c \lor d) \land ((\lnot a \land b) \lor \lnot (c \lor d)))$ [Start]
\\$ \lnot ((\lnot a \land b) (+) \lnot(c \lor d)) \equiv \lnot ((a \lor \lnot b \lor c \lor d) \land ((\lnot a \land b) \lor \lnot (c \lor d)))$ [Equivalence definition]
\\$  \lnot (\lnot ((\lnot a \land b) \land \lnot (c \lor d)) \land ((\lnot a \land b) \lor \lnot (c \lor d))) \equiv \lnot ((a \lor \lnot b \lor c \lor d) \land ((\lnot a \land b) \lor \lnot (c \lor d)))$ [Xor definition]
\\$  \lnot (\lnot ((\lnot a \land b) \land \lnot (c \lor d)) \land ((\lnot a \land b) \lor \lnot (c \lor d))) \equiv \lnot ((\lnot( \lnot a \land b) \lor c \lor d) \land ((\lnot a \land b) \lor \lnot (c \lor d)))$ [DM]
\\$  \lnot (\lnot ((\lnot a \land b) \land \lnot (c \lor d)) \land ((\lnot a \land b) \lor \lnot (c \lor d))) \equiv \lnot ((\lnot( \lnot a \land b) \lor (c \lor d)) \land ((\lnot a \land b) \lor \lnot (c \lor d)))$ [ASS]
\\$  \lnot (\lnot ((\lnot a \land b) \land \lnot (c \lor d)) \land ((\lnot a \land b) \lor \lnot (c \lor d))) \equiv \lnot (\lnot(( \lnot a \land b) \land \lnot (c \lor d)) \land ((\lnot a \land b) \lor \lnot (c \lor d)))$ [DM]
\\
\\3(c). C $\equiv$ F
\\ Truth table of C:
\\
\begin{displaymath}
\begin{array}{|c c c c|c|c|c|}
a & b & c & d & (a \land b \land c)&\lnot(b \land c \land d)&f \equiv (a \land b \land c)\lor \lnot(b \land c \land d)\\
\hline 
1&1&1&X&1&X&1\\
X&0&X&X&X&1&1\\
X&X&0&X&X&1&1\\
X&X&X&0&X&1&1\\
\end{array}
\end{displaymath}
\\ Since the statement is separated by an Or gate, we can find what makes either terms true and ignore the other term. That lets us deduce the truth table quickly. Unmentioned input combination(s) is false.
\\
\\ Truth table of F:
\\
\begin{displaymath}
\begin{array}{|c c c c|c|c|c|c|c|}
a & b & c & d & w \equiv (c (+) d) & x \equiv \lnot (a (+) b) & y \equiv \lnot (c \lor d) & z \equiv (a \land \lnot b) & f = w \lor x \lor y \lor z\\
\hline 
0 & 0 & 0 & 0 &0&1&1&0&1\\
0 & 0 & 0 & 1 &1&1&0&0&1\\
0 & 0 & 1 & 0 &1&1&0&0&1\\
0 & 0 & 1 & 1 &0&1&0&0&1\\
0 & 1 & 0 & 0 &0&0&1&0&1\\
0 & 1 & 0 & 1 &1&0&0&0&1\\
0 & 1 & 1 & 0 &1&0&0&0&1\\
0 & 1 & 1 & 1 &0&0&0&0&0\\
1 & 0 & 0 & 0 &0&0&1&1&1\\
1 & 0 & 0 & 1 &1&0&0&1&1\\
1 & 0 & 1 & 0 &1&0&0&1&1\\
1 & 0 & 1 & 1 &0&0&0&1&1\\
1 & 1 & 0 & 0 &0&1&1&0&1\\
1 & 1 & 0 & 1 &1&1&0&0&1\\
1 & 1 & 1 & 0 &1&1&0&0&1\\
1 & 1 & 1 & 1 &0&1&0&0&1\\
\end{array}
\end{displaymath}
\\
\\ These two truth tables' output values are the same for given input value (ie only 0111 input makes output 0), therefore the two logical statements are equivalent.
\newpage
\item[14] In this problem we will explore circuit design with limited resources by taking advantage of \textit {functionally complete} sets of logic gates. A set of logic gates is functionally complete if it is able to simulate all of the operations $\{AND, OR, NOT\}$. Here, we will show that $\{NOR\}$ alone is functionally complete.

For reference, a 2-input NOR operation on inputs $a$ and $b$ is written as $a \downarrow b$, and is equivalent to $\lnot (a \lor b)$.

\begin{enumerate}
    \item Using rules of logical equivalence, show that the 2-input NOR operation can be used to simulate a 1-input NOT operation.
    \vspace{1.0in}
    \item Using rules of logical equivalence, show that two 2-input NOR operations can be used to simulate a 2-input OR operation.
    \vspace{2.0in}
    \item Using a circuit diagram with accompanying truth table, show that multiple 2-input NOR gates (with non-inverted inputs) can be used to simulate a 2-input AND gate. Be sure to label your input and output signals.
    \vspace{2.5in}
    
    \newpage
    
    \item Below is a truth table for a system with 4 inputs. Design a circuit to implement the function, using only 2-input NOR gates with non-inverted inputs. A modest penalty will be applied to solutions using 16 or more gates. Hint: begin with a circuit using other gates, that can be easily converted to NOR gates using the double-negative law and DeMorgan's laws.
    
    \begin{tabular}{cccc|c}
      $x_3$ & $x_2$ & $x_1$ & $x_0$ & $f$ \\
      \hline
      F & F & F & F & F \\
      F & F & F & T & F \\
      F & F & T & F & T \\
      F & F & T & T & F \\
      \hline
      F & T & F & F & T \\
      F & T & F & T & F \\
      F & T & T & F & F \\
      F & T & T & T & F \\
      \hline
      T & F & F & F & F \\
      T & F & F & T & F \\
      T & F & T & F & T \\
      T & F & T & T & F \\
      \hline
      T & T & F & F & T \\
      T & T & F & T & T \\
      T & T & T & F & F \\
      T & T & T & T & F \\
    \end{tabular}
    
\end{enumerate}

\newpage
\newpage
\\3(a).
\\
\leavevmode\lower45pt\hbox{\includegraphics[scale=0.75]{3a.jpg}}
\\Proof:
\\ $ \equiv \lnot (a \lor a)$ [Start]
\\ $ \equiv \lnot (a)$ [ID]
\\ $ \equiv \lnot a$ [ASS]
\\ \boxed{}
\\
\\3(b).
\\
\leavevmode\lower45pt\hbox{\includegraphics[scale=0.75]{3b.jpg}}
\\ Proof:
\\ $ \equiv \lnot (\lnot (a \lor b) \lor \lnot (a \lor b))$ [Start]
\\ $ \equiv \lnot (\lnot (a \lor b))$ [ID]
\\ $ \equiv (a \lor b)$ [DNEG]
\\ \boxed{}
\\
\\3(c).
\\
\leavevmode\lower45pt\hbox{\includegraphics[scale=0.75]{3c.jpg}}
\begin{displaymath}
\begin{array}{|c c|c|c|c|c|}
a & b & \lnot (a \lor a) \equiv \lnot a & \lnot (b \lor b) \equiv \lnot b & \lnot (\lnot (a \lor a) \lor \lnot (b \lor b)) \equiv \lnot (\lnot a \lor \lnot b) & a \land b\\
\hline 
0 & 0 & 1 & 1 & 0 & 0\\
0 & 1 & 1 & 0 & 0 & 0\\
1 & 0 & 0 & 1 & 0 & 0\\
1 & 1 & 0 & 0 & 1 & 1\\
\end{array}
\end{displaymath}
\\3(d).
\\ Since the question didn't specify how we should simplify the truth table, I will use K-map for simplicity:
\begin{displaymath}
\begin{array}{|c|c|c|c|c|}
- & x1 \land x0 & \lnot x1 \land x0 & \lnot x1 \land \lnot x0 & x1 \land \lnot x0 \\
\hline 
x3 \land x2 & 0&1&1&0\\
\lnot x3 \land x2 & 0&0&1&0\\
\lnot x3 \land \lnot x2 & 0&0&0&1\\
x3 \land \lnot x2 & 0&0&0&1\\
\end{array}
\end{displaymath}
\\ Finding sum of minterms, I can get:
\\ $f \equiv (x2 \land \lnot x1 \land \lnot x0) \lor (x3 \land x2 \land \lnot x1) \lor (\lnot x2 \land x1 \land \lnot x0)$
\\
\\ I can use logical properties to change the logical statement into one that's made up of only Nor gates:
\\
\\ Proof:
\\ $\equiv (x2 \land \lnot x1 \land \lnot x0) \lor (x3 \land x2 \land \lnot x1) \lor (\lnot x2 \land x1 \land \lnot x0)$ [Start]
\\ $\equiv \lnot (\lnot x2 \lor x1 \lor x0) \lor \lnot (\lnot x3 \lor \lnot x2 \lor x1) \lor \lnot (x2 \lor \lnot x1 \lor x0)$ [DM to all And terms]
\\ $\equiv \lnot (\lnot (x2 \lor x2) \lor x1 \lor x0) \lor \lnot (\lnot (x3 \lor x3) \lor \lnot (x2 \lor x2) \lor x1) \lor \lnot (x2 \lor \lnot (x1 \lor x1) \lor x0)$ [ID to all Not terms]
\\ $\equiv \lnot (\lnot (\lnot (\lnot (x2 \lor x2) \lor x1 \lor x0) \lor \lnot (\lnot (x3 \lor x3) \lor \lnot (x2 \lor x2) \lor x1) \lor \lnot (x2 \lor \lnot (x1 \lor x1) \lor x0)))$ [DNEG]
\\ \boxed{}
\\
\\ This logical statement can be simplified further, but it already uses under 16 gates, as shown here:
\\ 
\leavevmode\lower45pt\hbox{\includegraphics[scale=0.75]{3d.jpg}}
\newpage
\item[8] Design  a circuit that  takes five inputs  $x_4$,  $x_3$, $x_2$, $x_1$,  $x_0$
  representing an unsigned binary number in the range of $\left[0,31\right]$ and a single
  output $z$ which is \texttt {true} if (and only if)  the binary number specified by the
  inputs is a prime number. Assume that 0 and 1 are not prime, and 2 is prime.
  For instance, your circuit should output \texttt {true} in the following cases:
  \begin{itemize}                                                                                   
  \item  $x_4 =  \texttt{false}$,  $x_3 =  \texttt{false}$, $x_2  =  \texttt{false}$, $x_1  =
    \texttt{true}$, $x_0 = \texttt{false}$ (decimal 2).
  \item  $x_4 =  \texttt{true}$,  $x_3 =  \texttt{false}$, $x_2  =  \texttt{false}$, $x_1  =
    \texttt{true}$, $x_0 = \texttt{true}$ (decimal 19).
  \item  $x_4 =  \texttt{true}$,  $x_3 =  \texttt{true}$, $x_2  =  \texttt{true}$, $x_1  =
    \texttt{false}$, $x_0 = \texttt{true}$ (decimal 29).
  \end{itemize}
  but it should output false in these cases:                                                        
  \begin{itemize}                                                                                   
  \item  $x_4 =  \texttt{false}$,  $x_3 =  \texttt{false}$, $x_2  =  \texttt{false}$, $x_1  =
    \texttt{false}$, $x_0 = \texttt{true}$ (decimal 1).
  \item  $x_4 =  \texttt{false}$,  $x_3 =  \texttt{true}$, $x_2  =  \texttt{true}$, $x_1  =
    \texttt{false}$, $x_0 = \texttt{false}$ (decimal 12).
  \item  $x_4 =  \texttt{true}$,  $x_3 =  \texttt{true}$, $x_2  =  \texttt{false}$, $x_1  =
    \texttt{false}$, $x_0 = \texttt{true}$ (decimal 25).
  \end{itemize}                                                                                     
  Justify your answer! You may use any gates available in Logisim, with any number of inputs
  and/or inverted inputs. A modest penalty will be applied to solutions using 11 or more gates.
  \newpage
\newpage
\\ 4.
\\ I know that in the range of [0, 31] the prime numbers are:  2, 3, 5, 7, 11, 13, 17, 19, 23, 29, 31. I can write a truth table from these outputs. 
\begin{displaymath}
\begin{array}{|c c c c c|c|c|}
x4&x3&x2&x1&x0&Decimal&z\\
\hline 
0 & 0 & 0 & 0 & 0 & 0&0\\
0 & 0 & 0 & 0 & 1 & 1&0\\
0 & 0 & 0 & 1 & 0 & 2&1\\
0 & 0 & 0 & 1 & 1 &3&1\\
0 & 0 & 1 & 0 & 0 &4&0\\
0 & 0 & 1 & 0 & 1 &5&1\\
0 & 0 & 1 & 1 & 0 &6&0\\
0 & 0 & 1 & 1 & 1 &7&1\\
0 & 1 & 0 & 0 & 0 &8&0\\
0 & 1 & 0 & 0 & 1 &9&0\\
0 & 1 & 0 & 1 & 0 &10&0\\
0 & 1 & 0 & 1 & 1 &11&1\\
0 & 1 & 1 & 0 & 0 &12&0\\
0 & 1 & 1 & 0 & 1 &13&1\\
0 & 1 & 1 & 1 & 0 &14&0\\
0 & 1 & 1 & 1 & 1 &15&0\\
1 & 0 & 0 & 0 & 0 &16&0\\
1 & 0 & 0 & 0 & 1 &17&1\\
1 & 0 & 0 & 1 & 0 &18&0\\
1 & 0 & 0 & 1 & 1 &19&1\\
1 & 0 & 1 & 0 & 0 &20&0\\
1 & 0 & 1 & 0 & 1 &21&0\\
1 & 0 & 1 & 1 & 0 &22&0\\
1 & 0 & 1 & 1 & 1 &23&1\\
1 & 1 & 0 & 0 & 0 &24&0\\
1 & 1 & 0 & 0 & 1 &25&0\\
1 & 1 & 0 & 1 & 0 &26&0\\
1 & 1 & 0 & 1 & 1 &27&0\\
1 & 1 & 1 & 0 & 0 &28&0\\
1 & 1 & 1 & 0 & 1 &29&1\\
1 & 1 & 1 & 1 & 0 &30&0\\
1 & 1 & 1 & 1 & 1 &31&1\\
\end{array}
\end{displaymath}
\\ Since the question did not specify the method, I can use K-map made from this truth table to get logical statement of $z$:
\\
\begin{displaymath}
\begin{array}{|c|c|c|c|c|}
- & x1 \land x0 & \lnot x1 \land x0 & \lnot x1 \land \lnot x0 & x1 \land \lnot x0 \\
\hline 
x4 \land x3 \land x2 & 1&1&0&0\\
x4 \land \lnot x3 \land x2 & 1&0&0&0\\
x4 \land \lnot x3 \land \lnot x2 & 1&1&0&0\\
x4 \land x3 \land \lnot x2 & 0&0&0&0\\
\lnot x4 \land x3 \land \lnot x2 & 1&0&0&0\\
\lnot x4 \land \lnot x3 \land \lnot x2 & 1&0&0&1\\
\lnot x4 \land \lnot x3 \land x2 & 1&1&0&0\\
\lnot x4 \land x3 \land x2 & 0&1&0&0\\
\end{array}
\end{displaymath}
\\
\\ We can rewrite these two K-maps into logical statements as follow:
\\ $ z = (x3 \land x2 \land \lnot x1 \land x0) \lor (\lnot x4 \land \lnot x3 \land x2 \land x0) \lor (\lnot x4 \land \lnot x3 \land \lnot x2 \land x1) \lor (\lnot x4 \land \lnot x2 \land x1 \land x0) \lor (x4 \land \lnot x3 \land \lnot x2 \land x0) \lor (x4 \land x2 \land x1 \land x0) $
\\
\\ Using K-map, we can be sure that this is the most simplified form of the logical statement. While there are one Xor and one Xnor simplifications we can make with $(4(+)2)$ and $\lnot (4(+)2)$, it doesn't decrease the number of logic gates since we still need to wrap the Xor and Xnor with an And gate (respectively) to pair it with other terms ($(\lnot x3 \land x0)$ and $(x1 \land x0)$, respectively). We can make this into a logic circuit:
\\
\leavevmode\lower45pt\hbox{\includegraphics[scale=0.75]{4.PNG}}
\\
\\ Note that I have to use two separate Or gates at the output since Logisim has 5 input maximum for their Or gate.
\\
\\ 
\end{question}
\end{document}