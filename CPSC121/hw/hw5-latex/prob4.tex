 \item[8] Using mathematical induction, prove that
 \begin{displaymath}
 \sum_{i=1}^n (i-1) 2^i = (n-2)2^{n+1}
 + 4
 \end{displaymath}
 
 \begin{Questions}
 \\{\color{NavyBlue}
 \\Proof:
 \\
 \\Part 1: Base case
 \begin{gather}
 (n-1) 2^n = (n-2)2^{n+1}+ 4
 \\(1-1) 2^1 = (1-2)2^{1+1}+ 4
 \\0 = 0
 \end{gather}
 \\Part 2: Induction step
 \\
 \\ Assuming that n case is true, we have to prove that this statement is true: 
 \begin{gather}
 \sum_{i=1}^{n+1} (i-1) 2^i = ((n+1)-2)2^{(n+1)+1}+ 4
 \end{gather}
 \\To prove that [4] is true, we say that the sum of n+1 terms is equals to the sum of n terms plus the n+1th term, and simplify from there:
  \begin{gather}
 \sum_{i=1}^{n+1} (i-1) 2^i = \sum_{i=1}^n (i-1) 2^i + ((n+1)-1) 2^{n+1}
 \\ ((n+1)-2)2^{(n+1)+1}+ 4 = ((n-2)2^{n+1}
 + 4) + ((n+1)-1) 2^{n+1}
 \\ (n-1)2^{n+2}+ 4 = (n-2)2^{n+1}
  + n 2^{n+1} + 4
  \\ (n-1)2^{n+2}= (n-2)2^{n+1}
  + n 2^{n+1}
   \\ n2^{n+2} - 2^{n+2} = (n-2)2^{n+1}
  + n 2^{n+1}
   \\ n2^{n+2} - 2^{n+2} = n2^{n+1} - 2^{n+2}
  + n 2^{n+1}
    \\ n2^{n+2} - 2^{n+2} = n2^{n+1} - 2^{n+2}
  + n 2^{n+1}
     \\ n2^{n+2} - 2^{n+2} = 2n2^{n+1} - 2^{n+2}
     \\ n2^{n+2} - 2^{n+2} = n2^{n+2} - 2^{n+2}
 \end{gather}
 \\Steps 5-7 is substituting terms. Step 8 cancels 4 from both sides. Steps 9-10 expands (n-1) and (n-2) terms by distributive property. Steps 11-13 combine like terms to get the final equality. 
 \\Thus, we have proved that when n case is true, n+1 case is true as well. 
 \\}
 \begin{figure}
\begin{verbatim}
         _______                   _____                    _____          
        /::\    \                 /\    \                  /\    \         
       /::::\    \               /::\    \                /::\    \        
      /::::::\    \             /::::\    \              /::::\    \       
     /::::::::\    \           /::::::\    \            /::::::\    \      
    /:::/~~\:::\    \         /:::/\:::\    \          /:::/\:::\    \     
   /:::/    \:::\    \       /:::/__\:::\    \        /:::/  \:::\    \    
  /:::/    / \:::\    \     /::::\   \:::\    \      /:::/    \:::\    \   
 /:::/____/   \:::\____\   /::::::\   \:::\    \    /:::/    / \:::\    \  
|:::|    |     |:::|    | /:::/\:::\   \:::\    \  /:::/    /   \:::\ ___\ 
|:::|____|     |:::|____|/:::/__\:::\   \:::\____\/:::/____/     \:::|    |
 \:::\   _\___/:::/    / \:::\   \:::\   \::/    /\:::\    \     /:::|____|
  \:::\ |::| /:::/    /   \:::\   \:::\   \/____/  \:::\    \   /:::/    / 
   \:::\|::|/:::/    /     \:::\   \:::\    \       \:::\    \ /:::/    /  
    \::::::::::/    /       \:::\   \:::\____\       \:::\    /:::/    /   
     \::::::::/    /         \:::\   \::/    /        \:::\  /:::/    /    
      \::::::/    /           \:::\   \/____/          \:::\/:::/    /     
       \::::/____/             \:::\    \               \::::::/    /      
        |::|    |               \:::\____\               \::::/    /       
        |::|____|                \::/    /                \::/____/        
         ~~                       \/____/                  ~~                  
\end{verbatim}
\caption{A blessed QED to end a blessed proof}
\end{figure}

 \vfill\eject
 \clearpage
 \end{Questions}
 