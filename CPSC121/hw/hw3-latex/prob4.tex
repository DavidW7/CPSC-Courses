%
% Question 4
%
\item [9] Rewrite  each of the following  theorems using quantifiers and  predicates. Note that the theorems are not {\em precisely} stated. It is up to you to choose reasonable sets from which the variables should be drawn. Further, notice that the definitions we have given you should be stated as predicates, but we would like {\em you} to formalize their specification. Make sure that your theorems have no unbound variables. 


  \begin{enumerate}
  
%
% Question 4(a)
%
  \item[3]
    \textbf{Theorem}: For any non-zero real number $x$ there is exactly one $y$ such that $\frac{x}{y}=-1$.
  \vspace{2in}

%
% Question 4(b)
%
  \item [3] \textbf{Theorem}: For any positive integer $x \geq 2$, $x$ is prime if and only if $x$ is evenly divisible by 1 and itself, and no other values in between.
  \vspace{2in}

 
%
% Question 4(c)
%
  \item [3] We define "big-O" as follows:
  \[f(n)\in O(g(n)) : ~~\exists c\in\R^{+},\exists n_0\in\N, \forall n\in\N, n\geq n_0 \longrightarrow f(n)\leq cg(n)\]
  
  \textbf{Theorem}: Let $k$ be a constant. If $f(n)\in O(h(n))$ then $k\cdot f(n) \in O(h(n))$.
  \vspace{2.5in}
  
    \end{enumerate}



