\item[6] Given a finite set $S$ of $n$ integers, $S=\{s_1, s_2, \ldots, s_n\}$, consider the problem of finding a subset of those integers, such that the sum of the items in the subset is equal to 0.

One possible solution is to generate all possible subsets of $S$, and then calculate the sum of the values in each subset to see if it is equal to zero.

\begin{question}{a.}[3]
    \item[4] The number of possible subsets for a set of $n$ integers satisfies the following equation:
    \[
          B(n)= \begin{cases}
          2 B(n-1) & \textrm{if }n > 1\\
          1 & \textrm{if }n = 0\\
          \end{cases}
    \]
    For instance, when $n=0$ there is only one possible subset: the empty set.
    When $n = 2$ there are four possible subsets:
    \[
      \begin{array}{ll}
      \{  \} &
      \{ s_1 \} \\
      \{ s_2 \} &
      \{ s_1, s_2 \} \\
      \end{array}
    \]
    Give a (very  simple) mathematical expression for $B(n)$. Use the formula from the first part to justify your answer.
    \begin{Questions}
    \vfill
    \end{Questions}
    
    \item[2] Suppose you run an algorithm which takes a number $n$ and a set $S$ of that size and then outputs every subset on a machine where your algorithm takes $.0001$ seconds for each subset it outputs. 
    
    Fill in the following table with the largest value of $n$ for which your algorithm terminates for each duration indicated.\bigskip
    
    \setlength{\extrarowheight}{0.3cm}
    \begin{tabular}{|l|c||l|c|}\hline
    Duration & Largest value of $n$ & Duration & Largest value of n \\ \hline
    1 second &  &
    1 minute & \\ \hline
    1 hour & &
    1 day & \\ \hline
    1 month & &
    1 year & \\ \hline 
    10 years & &
    100 years & \\ \hline
    \end{tabular}
    
\end{question}

