\documentclass[a4paper, 20pt]{article}
\usepackage[utf8]{inputenc}
\usepackage{paralist}
\usepackage{jeffe}
\usepackage{handout}
\usepackage{tikz}
\usepackage{gensymb}
\usetikzlibrary{arrows,shapes.gates.logic.US,shapes.gates.logic.IEC,calc}
\usetikzlibrary{positioning}
\usepackage{hyperref}
\usepackage{aurical}
\usepackage[T1]{fontenc}
% definitions for formatting code blocks
\usepackage{listings}

\usepackage{karnaugh-map}

\def\mor{\mathbin{\mid}}
\def\mand{\mathbin{\char`\&}}
\def\rshift{\mathbin{\char`\>\char`\>}}

\def\lnot{\mathop{\sim}}
\def\implies{\mathop{\rightarrow}}
\def\xor{\mathop{\oplus}}

\def\ith#1{${#1}^{\textrm{\scriptsize th}}$}

\begin{document}
\headers{CPSC 121}{ }{Summer 2019}

\begin{center}
    \LARGE
    \textbf{HW 2}
    \\[1ex]
    \Large Due: 23:00, Wednesday May 22, 2019\\
\end{center}
    \LARGE
\begin{tabular}{rl}
 & \\
CS ID 1: &  d6a2b\\
 & \\
CS ID 2: & b9i2b\\
 & \\
\end{tabular}
\large

\textbf{Instructions:}
\begin{enumerate}
\item Do not change the problem statements we are giving you. Simply add your solutions by editing this latex document. 
\item If you need more space, add a page between the existing pages using the \texttt{\textbackslash newpage} command.
\item Include formatting to clearly distinguish your solutions from the given problem text (e.g. use a different font colour for your solutions). Improperly or insufficiently typeset submissions will receive a penalty.
\item Export the completed assignment as a PDF file for upload to Gradescope.
\item On Gradescope, upload only \textbf{one} copy per partnership. (Instructions for uploading to Gradescope are posted on the assignments page of the course website.)
\item Late submissions will be accepted up to 24 hours past the deadline with a penalty of 20\% of the assignment’s maximum value
\end{enumerate}

\textbf{Academic Conduct:} 
I certify that my assignment follows the academic conduct rules for of CPSC 121 as outlined on the course website. As part of those rules, when collaborating with anyone outside my group, (1) I and my collaborators took no record but names away, and (2) after a suitable break, my group created the assignment I am submitting without help from anyone other than the course staff. \\

\textbf{Version history:}
\begin{itemize}
    \item 2019-05-15 00:15 -- Initial version for release
\end{itemize}

\newpage
\begin{question}{1.}
\def\StartsWith#1#2{\textit{Starts}({#1},{#2})}

\item[9] Consider the following sets and predicates: 
  \begin{itemize}
  \item $D$: video game developers
  \item $G$: video game titles
  \item $P$: hardware platforms
  \item $M(d, g)$: developer $d$ made game $g$.
  \item $R(g, p)$: game $g$ released on platform $p$.
  \item $B(d, g)$: developer $d$ bribed a journalist to write a favourable review for game $g$.
  \end{itemize}

  Rewrite each of  the following statements using \textbf{only}  the quantifiers $\forall$
  and $\exists$, the predicates $M$, $R$ and $B$, the domains $D$, $G$, $P$, and $\mathbb{R}$ (the set of real numbers),
  logical connectives, and the operators $=$, $\neq$. $<$, $\le$, $\ge$ and $>$.
  
  If you feel a  sentence is ambiguous, then state your assumed interpretation.
  \begin{question}{a.}[2]

  \item[3] Every developer made a game that released on every platform.
   
   \vspace{2.0in}
   
  \item[3] Some developer bribed a journalist to write a favourable review for a game made by a different developer.
  
  \vspace{2.0in}
   
  \item[3] Some developer bribed journalists to write favourable reviews for every game that the developer made.
  
  \vspace{2.0in}

  \end{question}

\newpage
\\ 1(a).
\\ Proof:
\\ $ \equiv (\lnot p \land (q \rightarrow p)) \rightarrow \lnot q$ [Start]
\\ $ \equiv (\lnot p \land (\lnot q \lor p)) \rightarrow \lnot q$ [IMP]
\\ $ \equiv \lnot (\lnot p \land (\lnot q \lor p)) \lor \lnot q$ [IMP]
\\ $ \equiv ( p \lor \lnot (\lnot q \lor p)) \lor \lnot q$ [IMP]
\\ $ \equiv p \lor \lnot (\lnot q \lor p) \lor \lnot q$ [ASS]
\\ $ \equiv \lnot q \lor p \lor \lnot (\lnot q \lor p) $ [COM]
\\ $ \equiv (\lnot q \lor p) \lor \lnot (\lnot q \lor p) $ [ASS]
\\ $ \equiv T $ [NEG]
\\ \boxed{}
\\ This is a tautology.
\\
\\1(b).
\\Proof:
\\ $\equiv \lnot(a \land b) \rightarrow (b \rightarrow (a \oplus b))$ [Start]
\\ $\equiv (a \land b) \lor (b \rightarrow (a \oplus b))$ [IMP]
\\ $\equiv (a \land b) \lor (\lnot b \lor (a \oplus b))$ [IMP]
\\ $\equiv (a \land b) \lor (\lnot b \lor (\lnot a \land b) \lor (a \land \lnot b))$  [XOR]
\\ $\equiv (a \land b) \lor (\lnot b \lor (a \land \lnot b) \lor (\lnot a \land b))$  [COM]
\\ $\equiv (a \land b) \lor (\lnot b \lor (\lnot a \land b))$  [ABS]
\\ $\equiv (a \land b) \lor (\lnot b \lor \lnot a \land \lnot b \lor b)$ [DIST]
\\ $\equiv (a \land b) \lor (\lnot b \lor \lnot a \land T)$  [NEG]
\\ $\equiv (a \land b) \lor (\lnot b \lor \lnot a)$  [I]
\\ $\equiv (a \land b) \lor \lnot  (a \land b)$  [DM]
\\ $\equiv T$  [NEG]
\\\boxed{}
\\ This is a tautology.
\\
\\ 1(c).
\\ Proof:
\\ $\equiv \lnot c \rightarrow ((\lnot b \rightarrow c)\land (c \rightarrow \lnot a))$ [Start]
\\ $\equiv \lnot c \rightarrow (( b \lor c)\land (c \rightarrow \lnot a))$ [IMP]
\\ $\equiv \lnot c \rightarrow (( b \lor c)\land (\lnot c \lor \lnot a))$ [IMP]
\\ $\equiv c \lor (( b \lor c)\land (\lnot c \lor \lnot a))$ [IMP]
\\ $\equiv (c \lor b \lor c)\land (c \lor \lnot c \lor \lnot a)$ [DIST]
\\ $\equiv (c \lor c \lor b)\land (c \lor \lnot c \lor \lnot a)$ [COM]
\\ $\equiv (c \lor b)\land (c \lor \lnot c \lor \lnot a)$ [ID]
\iffalse
\\ $\equiv (c \lor b)\land (T \lor \lnot a)$ [NEG]
\\ $\equiv (c \lor b)\land \lnot a$ [UB]
\fi
\\\boxed{}
\\ This does not simplify to T. A sample case to prove it's not a tautology is when b is 0 and c is 0. This is not a tautology.
\\
\\ 1(d).
\\ Proof:
\\ $\equiv ((a \land c) \rightarrow (b \rightarrow c)) \land (c \rightarrow a)$ [Start]
\\ $\equiv ((a \land c) \rightarrow (b \rightarrow c)) \land (\lnot c \lor a)$ [IMP]
\iffalse
\\ $\equiv ((a \land c) \rightarrow (\lnot b \lor c)) \land (\lnot c \lor a)$ [IMP]
\\ $\equiv (\lnot (a \land c) \lor (\lnot b \lor c)) \land (\lnot c \lor a)$ [IMP]
\\ $\equiv (\lnot a \lor \lnot c \lor (\lnot b \lor c)) \land (\lnot c \lor a)$ [DM]
\fi
\\ \boxed{}
\\ This does not simplify to T. A sample case to prove it's not a tautology is when a is 0 and c is 1. This is not a tautology.
\\
\newpage
%
% Question 2
%
\item[9] 

Using the definitions given in question 1, translate each of the
  following predicate logic statements into English. Try to make your English translations
  as natural sounding as possible.

  \begin{question}{a.}[3]
  \item[3] $\exists d \in D, \forall x,y \in G, \exists p_1, p_2 \in P, M(d,x) \land M(d,y) \land R(x,p_1) \land R(y,p_2) \implies p_1 = p_2 \implies \lnot(\exists g \in G, M(d,g) \land B(d,g))$.
  \begin{Questions}
    \vfill
  \end{Questions}
  
  \item[3] $\exists d \in D, \forall g \in G, \lnot M(d,g) \land (R(g, \text{Nintendo Switch}) \implies B(d,g))$.
  \begin{Questions}
    \vfill
  \end{Questions}

  \item[3] $\exists d \in D, \forall p \in P, \exists  g_1, g_2 \in G, M(d, g_1) \land M(d, g_2) \land R(g_1, p) \land \lnot R(g_2, p)$.
  \begin{Questions}
    \vfill
  \end{Questions}
  
  \end{question}
\newpage
\\ 2(a).
\\ $30924_1_0 = 0111 1000 1100 1100_2$
\\
\\ 2(b).
\\ 30924 mod 32 = 12
\\
\\ 2(c).
\\ $12_1_0 = 01100_2$
\\
\\ 2(d).
\\ 12 in binary appears in 5 least significant digits of 30924 in binary.
\\
\\ 2(e).
\\ I know that 30924 mod 32 = 12, and 16 is a common denominator of 32, so x mod 32 must output x mod 16, or a number in between 16 to 31. In this case, the answer to 30924 mod 32 = 12, so 30924 mod 16 must also = 12.
\\
\\ 2(f).
\\ $30924_1_0 = 0111 1000 1100 1100_2$
\\ $48701_1_0 = 1011 1110 0011 1101_2$
\\ Adding only the 5 least significant digits:
\\ 01100 + 11101 = 101001
\\ Five least significant digits of a + b in binary is 01001.
\\
\\2(g).
\\ 30924 mod 32 = 12
\\ 8 mod 12 = 8
\\ 12*8 = 96
\\ 96 mod 32 = 0
\\ 30924·8 mod 32 = 0
\\
\\2(h).
\\ $-30924_1_0 = 1000 0111 0011 0100_2$
\\ 
\\ 2(i).
\\ -30924 mod 32 = 20 = $10100_2$
\\
\\ 2(j).
\\ They are both 5 least significant digits of the number that 32 mod into.
\\
\\ 2(k).
\\ $-30924_1_0 = 1000 0111 0011 0100_2$
\\ $16653_1_0 = 0100 0001 0000 1101_2$
\\ Adding only the 5 least significant digits:
\\ 10100 + 01101 = 100001
\\ Five least significant digits of a + b in binary is 00001.
\\
\\2(l).
\\ -30924 mod 32 = 20
\\ 8 mod 12 = 8
\\ 20*8 = 160
\\ 160 mod 32 = 0
\\−30924·8 mod 32 = 0
\\
\\2(m).
\\ 3 = -32 mod 5
\\ 4 = 9 mod 5
\\ -32/9 = -3.556
\\ -3.556 mod 5 = 1.444
\\ 3/4 = 0.75
\\ 1.44 $\neq$ 0.75
\\ Modular operation rule can't extend to division, proven by counter example above.
\newpage
\item[14] In this problem we will explore circuit design with limited resources by taking advantage of \textit {functionally complete} sets of logic gates. A set of logic gates is functionally complete if it is able to simulate all of the operations $\{AND, OR, NOT\}$. Here, we will show that $\{NOR\}$ alone is functionally complete.

For reference, a 2-input NOR operation on inputs $a$ and $b$ is written as $a \downarrow b$, and is equivalent to $\lnot (a \lor b)$.

\begin{enumerate}
    \item Using rules of logical equivalence, show that the 2-input NOR operation can be used to simulate a 1-input NOT operation.
    \vspace{1.0in}
    \item Using rules of logical equivalence, show that two 2-input NOR operations can be used to simulate a 2-input OR operation.
    \vspace{2.0in}
    \item Using a circuit diagram with accompanying truth table, show that multiple 2-input NOR gates (with non-inverted inputs) can be used to simulate a 2-input AND gate. Be sure to label your input and output signals.
    \vspace{2.5in}
    
    \newpage
    
    \item Below is a truth table for a system with 4 inputs. Design a circuit to implement the function, using only 2-input NOR gates with non-inverted inputs. A modest penalty will be applied to solutions using 16 or more gates. Hint: begin with a circuit using other gates, that can be easily converted to NOR gates using the double-negative law and DeMorgan's laws.
    
    \begin{tabular}{cccc|c}
      $x_3$ & $x_2$ & $x_1$ & $x_0$ & $f$ \\
      \hline
      F & F & F & F & F \\
      F & F & F & T & F \\
      F & F & T & F & T \\
      F & F & T & T & F \\
      \hline
      F & T & F & F & T \\
      F & T & F & T & F \\
      F & T & T & F & F \\
      F & T & T & T & F \\
      \hline
      T & F & F & F & F \\
      T & F & F & T & F \\
      T & F & T & F & T \\
      T & F & T & T & F \\
      \hline
      T & T & F & F & T \\
      T & T & F & T & T \\
      T & T & T & F & F \\
      T & T & T & T & F \\
    \end{tabular}
    
\end{enumerate}

\newpage
\newpage
\\3(a).
\\ a = "There is a chance of rain"
\\ b = "Geoff ate bean chili for lunch"
\\ c = "Geoff will ride his bicycle"
\\ d = "Geoff washed his car in the morning"
\\
\\ Argument:
\\$[1] (a \lor b) \rightarrow \lnot c$
\\$[2] \lnot d \rightarrow \lnot a$
\\$[3] \lnot d \land \lnot b$
\\Therefore [Conclusion] c
\\
\\ Proof:
\\$[5] \lnot (a \lor b) \lor \lnot c$ [IMP 1]
\\$[6] \lnot b$ [SPEC 3]
\\$[7] \lnot d$ [SPEC 3]
\\$[8] \lnot a$ [M.PON 2, 7]
\\$[9] \lnot a \land \lnot b$ [CONJ 6, 8]
\\$[10] a \lor b$ [DM 9]
\\$[11] \lnot c$ [M.PON 1, 10] 
\\\boxed{}
\\
\\ Argument is not valid since premises are true, but conclusion is false.
\\
\\3(b).
\\ a = "Geoff works hard enough"
\\ b = "Geoff gets fired"
\\ c = "Geoff gets paid"
\\ d = "Geoff buys food to eat"
\\
\\Argument:
\\$[1] a \land \lnot b \rightarrow c$
\\$[2] c \rightarrow d$
\\$[3] \lnot d$
\\ Therefore [Conclusion] $\lnot a (+) b$
\\
\\Proof:
\\$[4] a \land \lnot b \rightarrow d$ [TRANS 1, 2]
\\$[5] \lnot (a \land \lnot b)$ [T.MOL 3, 4]
\\$[6] \lnot a \lor b$ [DM 5]
\\ \boxed{}
\\
\\ Argument is invalid since there's at least one truth assignment where all the premises are true but the conclusion is false. A sample one is:
\\ a = 0, b = 1, c = X, d = X
\\
\\3(c).
\\ a = "30,000 cookies disappeared from Christie’s factor"
\\ b = "Christie destroyed the cookie"
\\ c = "Keebler stole the cookie"
\\ d = "a tree-dwelling elf became the new CEO of Christie"
\\ e = "the new Christie CEO colluded with Keebler"
\\ f = "new Christie CEO have a secret meeting with Keebler’s industrial spies"
\\
\\Argument:
\\ $[1] a \rightarrow b \lor c$
\\ $[2] \lnot (d \land e) \rightarrow a$
\\ $[3] \lnot f \rightarrow \lnot c$
\\ $[4] \lnot b$
\\ $[5] f$
\\ Therefore [Conclusion] $\lnot e$
\\
\\ Argument is invalid since there's at least one truth assignment where all the premises are true but the conclusion is false. A sample one is:
\\ a = 1, b = 0, c = 1, d = 0, e = 1, f = 1
\\
\\3(d).
\\This argument is valid.
\\
\\ Argument:\\
$[1]\lnot p \land q$\\
$[2]r \rightarrow p$\\
$[3]\lnot r \rightarrow (s \land t)$\\
$[4]s \rightarrow (t \lor p)$\\
$\therefore [Conclusion] t  $\\
\\ Proof:
\\ $[5] \lnot p$ [SPEC 1]
\\ $[6] \lnot r$ [M.TOL 2,5]
\\ $[7] (s \land t)$ [M.PON 3, 6]
\\ $[8] t$ [SPEC 7]
\\ \boxed{}
\\
\\ Argument is valid because conclusion and premises are true.
\newpage
\item[8] Design  a circuit that  takes five inputs  $x_4$,  $x_3$, $x_2$, $x_1$,  $x_0$
  representing an unsigned binary number in the range of $\left[0,31\right]$ and a single
  output $z$ which is \texttt {true} if (and only if)  the binary number specified by the
  inputs is a prime number. Assume that 0 and 1 are not prime, and 2 is prime.
  For instance, your circuit should output \texttt {true} in the following cases:
  \begin{itemize}                                                                                   
  \item  $x_4 =  \texttt{false}$,  $x_3 =  \texttt{false}$, $x_2  =  \texttt{false}$, $x_1  =
    \texttt{true}$, $x_0 = \texttt{false}$ (decimal 2).
  \item  $x_4 =  \texttt{true}$,  $x_3 =  \texttt{false}$, $x_2  =  \texttt{false}$, $x_1  =
    \texttt{true}$, $x_0 = \texttt{true}$ (decimal 19).
  \item  $x_4 =  \texttt{true}$,  $x_3 =  \texttt{true}$, $x_2  =  \texttt{true}$, $x_1  =
    \texttt{false}$, $x_0 = \texttt{true}$ (decimal 29).
  \end{itemize}
  but it should output false in these cases:                                                        
  \begin{itemize}                                                                                   
  \item  $x_4 =  \texttt{false}$,  $x_3 =  \texttt{false}$, $x_2  =  \texttt{false}$, $x_1  =
    \texttt{false}$, $x_0 = \texttt{true}$ (decimal 1).
  \item  $x_4 =  \texttt{false}$,  $x_3 =  \texttt{true}$, $x_2  =  \texttt{true}$, $x_1  =
    \texttt{false}$, $x_0 = \texttt{false}$ (decimal 12).
  \item  $x_4 =  \texttt{true}$,  $x_3 =  \texttt{true}$, $x_2  =  \texttt{false}$, $x_1  =
    \texttt{false}$, $x_0 = \texttt{true}$ (decimal 25).
  \end{itemize}                                                                                     
  Justify your answer! You may use any gates available in Logisim, with any number of inputs
  and/or inverted inputs. A modest penalty will be applied to solutions using 11 or more gates.
  \newpage
\newpage
\\4(a).
\\ Assuming carry lower = 1 if result from bits i-1...0 is A < B, and carry greater = 1 if result from bits i-1...0 is A > B. cl and cg can't be both 1 at once, so those values can be considered as don't-care terms. This assumption is in addition to the givens, where $gt=1$ if $A>B$, $eq=1$ if $A=B$, $lt=1$ if $A<B$.
\\
\\
\begin{tabular}{c c c c | c c c}
    $a$ & $b$ & $cl$ & $cg$ & $gt$ & $eq$ & $lt$ \\
    \hline
    0 & 0 & 0 & 0 & 0 & 1 & 0 \\
    0 & 0 & 0 & 1 & 1 & 0 & 0 \\
    0 & 0 & 1 & 0 & 0 & 0 & 1 \\
    0 & 0 & 1 & 1 & X & X & X \\
    \hline
    0 & 1 & 0 & 0 & 0 & 0 & 1 \\
    0 & 1 & 0 & 1 & 0 & 0 & 1 \\
    0 & 1 & 1 & 0 & 0 & 0 & 1 \\
    0 & 1 & 1 & 1 & X & X & X \\
    \hline
    1 & 0 & 0 & 0 & 1 & 0 & 0 \\
    1 & 0 & 0 & 1 & 1 & 0 & 0 \\
    1 & 0 & 1 & 0 & 1 & 0 & 0 \\
    1 & 0 & 1 & 1 & X & X & X \\
    \hline
    1 & 1 & 0 & 0 & 0 & 1 & 0 \\
    1 & 1 & 0 & 1 & 1 & 0 & 0 \\
    1 & 1 & 1 & 0 & 0 & 0 & 1 \\
    1 & 1 & 1 & 1 & X & X & X \\
  \end{tabular}
  
  \\
\\4(b).
\\ Without doing a K-map or a logic simplification, you can just look at it and see that gt and lt is a mux output using a xor b as the selector. When a xor b is 1, gt and lt takes a and b, respectively. When a xor b is 0, gt and lt takes cg and cl, respectively. We can use mux equivalence in logic gates to wire up gt and lt. eq is just a simple sum of minterms. 
\\
 \includegraphics[height=10cm]{4b.png} \\
 \\
\\ 4(c).
\\ It's connecting the 4 corresponding bits of A and B to each comparator, and connecting gt and lt of comparator i-1 to cg and cl of comparator i. cg and cl of comparator 0 should be grounded since it's the first digit / no carry yet. Final output is hooked to the 4th comparator (comparator 3). You can keep adding comparators to make your input however many digits you'd like.
\\
 \includegraphics[height=12cm]{4c.png} \\
\newpage
\end{question}
\end{document}
