\item[10] Later in the semester we will discover strategies for proving that for integers $a$, $b$, $c$, $d$, and $m$, if $a\equiv b\bmod{m}$ and $c\equiv d\bmod{m}$, then $a\cdot c\equiv b\cdot d\bmod{m}$, and $a+c\equiv b+d\bmod{m}$. In this problem, we will  investigate the meaning of these statements, and see their implications for number representation. Feel free to use a calculator for parts (a), (b), (f), (g), (h), (i) and (l).

\begin{question}{a.}[0.5]
    \item[0.5] Write the 16-bit unsigned representations of 30924.
    \begin{Questions}
    \vfill
    \end{Questions}
    \item[0.5] What is $30924\bmod 32$, expressed in decimal?
    \begin{Questions}
    \vfill
    \end{Questions}
    \item[0.5] Write the 5-bit unsigned representation of your answer to part (b).
    \begin{Questions}
    \vfill
    \end{Questions}
    \item[0.5] Describe the relationship between your answers to parts (a) and (c).
    \begin{Questions}
    \vfill
    \end{Questions}
    \item[1] Explain how you can compute the remainder when 30924 is divided by 16 (in decimal) \emph{in 10 seconds or less}.
    \begin{Questions}
    \vfill\vfill\eject
    \end{Questions}
    \item[0.5] If $a = 30924$, and $b= 48701$, find the least significant 5 bits of $a+b$ (in binary).
    \begin{Questions}
    \vfill
    \end{Questions}
    \item[0.5] Find $30924 \cdot 8\bmod{32}$ (in decimal).
    \begin{Questions}
    \vfill
    \end{Questions}
    \item[1]
    Write the 16-bit two's complement {\em signed} representation of $-30924$.
    \begin{Questions}
    \vfill
    \end{Questions}
    \item[1]
    Find $-30924\mod{32}$ (in binary).
    \begin{Questions}
    \vfill
    \end{Questions}
    \item[1]
    Describe the relationship between your answers to parts (c) and (i).
    \begin{Questions}
    \vfill\vfill
    \end{Questions}
    \item[0.5] If $a = -30924$, and $b= 16653$, find the least significant 5 bits of $a+b$ (in binary).
    \begin{Questions}
    \vfill
    \end{Questions}
    \item[0.5] Find $-30924 \cdot 8\mod{32}$ (in decimal).
    \begin{Questions}
    \vfill\eject
    \end{Questions}
    \item[2]
    By exploration, we have illustrated some of the implications of the simple rules of modular arithmetic. Show that the rule cannot be extended to the division operation, by finding a {\em counter example}. That is, find unsigned integers $a$, $b$, $c$, $d$, and $m$, so that $a\equiv b\mod{m}$ and $c\equiv d\mod{m}$, but $a/c\not\equiv b/d\mod{m}$.
    \begin{Questions}
    \vfill
    \end{Questions}
    
\end{question}
\vfill\eject
