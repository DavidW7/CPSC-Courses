\item[17] Logical Arguments
\begin{question}{a.}[5]

\item[4] Consider the following argument:
\begin{quote}
If there is a chance of rain, or he ate bean chili for lunch, Geoff will not ride his bicycle. Unless Geoff washed his car in the morning, there is no chance of rain. Today Geoff neither washed his car nor ate bean chili. Therefore, Geoff will ride his bicycle today.
\end{quote}

First represent this argument symbolically. Then determine whether it is valid or not.
Justify your answer.
\begin{Questions}
\vfill\eject
\end{Questions}

\item[4] Consider the following argument:
\begin{quote}
If Geoff works hard enough and does not get fired, then he will get paid. If he gets paid, then he will buy food to eat. Geoff has not bought food to eat. Therefore either Geoff did not work hard enough, or he got fired.
\end{quote}

First represent this argument symbolically. Then determine whether it is valid or not.
Justify your answer.
\begin{Questions}
\vfill\eject
\end{Questions}

\item [5]
Consider the following argument:
\begin{quote}
If 30,000 cookies disappeared from Christie's factory, then either Christie destroyed the cookies, or Keebler stole the cookies. If it is not the case that a tree-dwelling elf became the new CEO of Christie and the new Christie CEO colluded with Keebler, then 30,000 cookies disappeared from Christie's factory. If the new Christie CEO did not have a secret meeting with Keebler's industrial spies, then Keebler did not steal the cookies. Christie did not destroy the cookies, and the new Christie CEO had a secret meeting with Keebler's spies. Therefore, Christie's new CEO did not collude with Keebler! No collusion!
\end{quote}

First represent this argument symbolically. Then determine whether it is valid or not.
Justify your answer.
\begin{Questions}
\vfill\eject
\end{Questions}

\item[4]
Determine whether the following argument is valid. If it is valid, show a formal proof and if it is invalid provide a counter-example, i.e. an assignment of truth values that demonstrate a contradiction.


This argument is \underline{~~~~~~~~~~~~~~~~~~~~~}. (Fill in the blank with ``valid'' or ``invalid''.)
\vspace{0.2in}

\hspace{0.4in}\begin{tabular}{r}

$\lnot p \land q$\\
$r \rightarrow p$\\
$\lnot r \rightarrow (s \land t)$\\
$s \rightarrow (t \lor p)$\\
\hline
$\therefore  t  $\\
\end{tabular}
\vspace{0.2in}

Proof:

\hspace{0.4in}\begin{tabular}{r|l}
~~~~~~inference~~~~~~ & rule \\
\hline
 & ~~~~~~~~~~~~~~~~~~~~~~~\\
\hline
 & ~~~~~~~~~~~~~~~~~~~~~~~\\
\hline
 & ~~~~~~~~~~~~~~~~~~~~~~~\\
\hline
 & ~~~~~~~~~~~~~~~~~~~~~~~\\
\hline
 & ~~~~~~~~~~~~~~~~~~~~~~~\\
\hline
 & ~~~~~~~~~~~~~~~~~~~~~~~\\
\hline
 & ~~~~~~~~~~~~~~~~~~~~~~~\\
\hline
 & ~~~~~~~~~~~~~~~~~~~~~~~\\
\hline
 & ~~~~~~~~~~~~~~~~~~~~~~~\\
\hline
 & ~~~~~~~~~~~~~~~~~~~~~~~\\
\hline
 & ~~~~~~~~~~~~~~~~~~~~~~~\\
\hline
 & ~~~~~~~~~~~~~~~~~~~~~~~\\
\hline
 & ~~~~~~~~~~~~~~~~~~~~~~~\\
\hline
 & ~~~~~~~~~~~~~~~~~~~~~~~\\
\hline
 & ~~~~~~~~~~~~~~~~~~~~~~~\\
\hline
\end{tabular}
\large
\vspace{0.2in}

Invalidating truth assignment:\bigskip

$r=~~~~~~$, $s=~~~~~~$, $t=~~~~~~$, $u=~~~~~~$, $w=~~~~~~$


\end{question}
\newpage
