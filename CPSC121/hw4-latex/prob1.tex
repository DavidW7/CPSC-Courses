\item[9] Each of the  following theorems is either valid or invalid,  but the proof given
  is incorrect even if the theorem is valid. Explain briefly what mistake was made in each
  case.
  \begin{question}{a.}[4]
  \item[3] \textbf{Theorem}:  Let $f$, $g$  and $h$  be three functions  from $\mathbf{N}$
    into $\mathbf{R}^+$. If $f \in O(g)$ and $g \in O(h)$, then $f \in O(h)$.

    \noindent\textbf{Proof}: Consider  three unspecified functions  $f$, $g$ and  $h$ from
    $\mathbf{N}$ into $\mathbf{R}^+$. Assume that $f \in  O(g)$ and $g \in O(h)$. Since $f
    \in O(g)$, there is a real number $c$ and a positive integer $n_0$ such that for every
    $n \ge  n_0$, $f(n) \le cg(n)$.  Similarly, for every  $n \ge n_0$, $g(n)  \le ch(n)$.
    Therefore, for every $n  \ge n_0$, $f(n) \le cg(n) \le c(ch(n)) =  c^2 h(n)$. Hence $f
    \in O(h)$ using the constants $c^2$ and $n_0$.
    \begin{Questions}
    \\
    \\{\color{NavyBlue} The proof is wrong because the variable c in $f(n) \le cg(n)$ and $g(n) \le ch(n)$ are not the same variable, but rather a made up generalized variable for defining big-O formula. It would not end up as c^2, but more like $f(n) \le c_1g(n)$, $g(n) \le c_2h(n)$, and the final c being $c_1 * c_2$.}
    \\ 
      \vfill
    \end{Questions}

  \item[3] \textbf{Theorem}: If $n^2 + n - 6 \ge 0$, then $n \ge 2$.

    \noindent\textbf{Proof}: When $n \ge 2$, we know that $n^2 \ge 4$, so $n^2 + n \ge 6$,
    and therefore $n^2 + n - 6 \ge 0$.
    \begin{Questions}
    \\
    \\{\color{NavyBlue}The proof is wrong because the structure of the proof is wrong. The theorem is in the form of P->Q, where P=$n^2 + n - 6 \ge 0$, and Q=$n \ge 2$. The proof should use the structure of Q'->P' or simply P->Q, as opposed to currently using Q->P structure.
    \\
    \\This results in overlooking cases where n is less than or equal to -2. When $n^2 \ge 4$, $n \ge 2$ is not necessarily true, because n could have larger magnitude than -2 for $n^2 \ge 4$ to be true. Therefore, for cases of n is less than or equal to -2, $n^2 + n \ge 6$ can be false (ex: n = -2, $n^2 + n = (-2)^2 + (-2) = 2 \le 6$).}
      \vfill 
      
    \end{Questions}

  \item[3] \textbf{Theorem}: No matter  how we choose an integer $n$, the  value $n^3 + n$
    will be even.

    \noindent\textbf{Proof}: We use  a proof by contradiction. Assume that  the theorem is
    false. That is, $n^3 + n$ is odd. This is  not true, as we can see by choosing $n = 2$
    ($2^3 +  2 = 10$,  which is even).  So we have found  a contradiction, which  means the
    theorem is true.
    \begin{Questions}
    \\
    {\color{NavyBlue} \\The proof is wrong, since for proof by contradiction, we have to use the contradicted statement to make assumptions that would contradict for all n, not find a single n that would proof the contradicted statement false.
    \\
    \\A correct version of this proof would've been more like: Suppose $n^3 + n = x$, where x is odd. We choose a general n to have properties of an even number. n is manipulated using $n^3 + n = x$ to get some odd x, where it is manipulated again to get n, where n now has properties of an odd number. Therefore, theorem is true because when the theorem is false, a contradiction happens (n can't be even and odd at the same time).}
    
      \vfill\eject
    \end{Questions}
  \end{question}
